% Options for packages loaded elsewhere
\PassOptionsToPackage{unicode}{hyperref}
\PassOptionsToPackage{hyphens}{url}
%
\documentclass[
]{article}
\usepackage{lmodern}
\usepackage{amssymb,amsmath}
\usepackage{ifxetex,ifluatex}
\ifnum 0\ifxetex 1\fi\ifluatex 1\fi=0 % if pdftex
  \usepackage[T1]{fontenc}
  \usepackage[utf8]{inputenc}
  \usepackage{textcomp} % provide euro and other symbols
\else % if luatex or xetex
  \usepackage{unicode-math}
  \defaultfontfeatures{Scale=MatchLowercase}
  \defaultfontfeatures[\rmfamily]{Ligatures=TeX,Scale=1}
\fi
% Use upquote if available, for straight quotes in verbatim environments
\IfFileExists{upquote.sty}{\usepackage{upquote}}{}
\IfFileExists{microtype.sty}{% use microtype if available
  \usepackage[]{microtype}
  \UseMicrotypeSet[protrusion]{basicmath} % disable protrusion for tt fonts
}{}
\makeatletter
\@ifundefined{KOMAClassName}{% if non-KOMA class
  \IfFileExists{parskip.sty}{%
    \usepackage{parskip}
  }{% else
    \setlength{\parindent}{0pt}
    \setlength{\parskip}{6pt plus 2pt minus 1pt}}
}{% if KOMA class
  \KOMAoptions{parskip=half}}
\makeatother
\usepackage{xcolor}
\IfFileExists{xurl.sty}{\usepackage{xurl}}{} % add URL line breaks if available
\IfFileExists{bookmark.sty}{\usepackage{bookmark}}{\usepackage{hyperref}}
\hypersetup{
  pdftitle={HW6.R},
  pdfauthor={Malik},
  hidelinks,
  pdfcreator={LaTeX via pandoc}}
\urlstyle{same} % disable monospaced font for URLs
\usepackage[margin=1in]{geometry}
\usepackage{color}
\usepackage{fancyvrb}
\newcommand{\VerbBar}{|}
\newcommand{\VERB}{\Verb[commandchars=\\\{\}]}
\DefineVerbatimEnvironment{Highlighting}{Verbatim}{commandchars=\\\{\}}
% Add ',fontsize=\small' for more characters per line
\usepackage{framed}
\definecolor{shadecolor}{RGB}{248,248,248}
\newenvironment{Shaded}{\begin{snugshade}}{\end{snugshade}}
\newcommand{\AlertTok}[1]{\textcolor[rgb]{0.94,0.16,0.16}{#1}}
\newcommand{\AnnotationTok}[1]{\textcolor[rgb]{0.56,0.35,0.01}{\textbf{\textit{#1}}}}
\newcommand{\AttributeTok}[1]{\textcolor[rgb]{0.77,0.63,0.00}{#1}}
\newcommand{\BaseNTok}[1]{\textcolor[rgb]{0.00,0.00,0.81}{#1}}
\newcommand{\BuiltInTok}[1]{#1}
\newcommand{\CharTok}[1]{\textcolor[rgb]{0.31,0.60,0.02}{#1}}
\newcommand{\CommentTok}[1]{\textcolor[rgb]{0.56,0.35,0.01}{\textit{#1}}}
\newcommand{\CommentVarTok}[1]{\textcolor[rgb]{0.56,0.35,0.01}{\textbf{\textit{#1}}}}
\newcommand{\ConstantTok}[1]{\textcolor[rgb]{0.00,0.00,0.00}{#1}}
\newcommand{\ControlFlowTok}[1]{\textcolor[rgb]{0.13,0.29,0.53}{\textbf{#1}}}
\newcommand{\DataTypeTok}[1]{\textcolor[rgb]{0.13,0.29,0.53}{#1}}
\newcommand{\DecValTok}[1]{\textcolor[rgb]{0.00,0.00,0.81}{#1}}
\newcommand{\DocumentationTok}[1]{\textcolor[rgb]{0.56,0.35,0.01}{\textbf{\textit{#1}}}}
\newcommand{\ErrorTok}[1]{\textcolor[rgb]{0.64,0.00,0.00}{\textbf{#1}}}
\newcommand{\ExtensionTok}[1]{#1}
\newcommand{\FloatTok}[1]{\textcolor[rgb]{0.00,0.00,0.81}{#1}}
\newcommand{\FunctionTok}[1]{\textcolor[rgb]{0.00,0.00,0.00}{#1}}
\newcommand{\ImportTok}[1]{#1}
\newcommand{\InformationTok}[1]{\textcolor[rgb]{0.56,0.35,0.01}{\textbf{\textit{#1}}}}
\newcommand{\KeywordTok}[1]{\textcolor[rgb]{0.13,0.29,0.53}{\textbf{#1}}}
\newcommand{\NormalTok}[1]{#1}
\newcommand{\OperatorTok}[1]{\textcolor[rgb]{0.81,0.36,0.00}{\textbf{#1}}}
\newcommand{\OtherTok}[1]{\textcolor[rgb]{0.56,0.35,0.01}{#1}}
\newcommand{\PreprocessorTok}[1]{\textcolor[rgb]{0.56,0.35,0.01}{\textit{#1}}}
\newcommand{\RegionMarkerTok}[1]{#1}
\newcommand{\SpecialCharTok}[1]{\textcolor[rgb]{0.00,0.00,0.00}{#1}}
\newcommand{\SpecialStringTok}[1]{\textcolor[rgb]{0.31,0.60,0.02}{#1}}
\newcommand{\StringTok}[1]{\textcolor[rgb]{0.31,0.60,0.02}{#1}}
\newcommand{\VariableTok}[1]{\textcolor[rgb]{0.00,0.00,0.00}{#1}}
\newcommand{\VerbatimStringTok}[1]{\textcolor[rgb]{0.31,0.60,0.02}{#1}}
\newcommand{\WarningTok}[1]{\textcolor[rgb]{0.56,0.35,0.01}{\textbf{\textit{#1}}}}
\usepackage{graphicx,grffile}
\makeatletter
\def\maxwidth{\ifdim\Gin@nat@width>\linewidth\linewidth\else\Gin@nat@width\fi}
\def\maxheight{\ifdim\Gin@nat@height>\textheight\textheight\else\Gin@nat@height\fi}
\makeatother
% Scale images if necessary, so that they will not overflow the page
% margins by default, and it is still possible to overwrite the defaults
% using explicit options in \includegraphics[width, height, ...]{}
\setkeys{Gin}{width=\maxwidth,height=\maxheight,keepaspectratio}
% Set default figure placement to htbp
\makeatletter
\def\fps@figure{htbp}
\makeatother
\setlength{\emergencystretch}{3em} % prevent overfull lines
\providecommand{\tightlist}{%
  \setlength{\itemsep}{0pt}\setlength{\parskip}{0pt}}
\setcounter{secnumdepth}{-\maxdimen} % remove section numbering

\title{HW6.R}
\author{Malik}
\date{2020-10-19}

\begin{document}
\maketitle

\begin{Shaded}
\begin{Highlighting}[]
\CommentTok{###################################################################################}
\CommentTok{#}
\CommentTok{# HW6.R}
\CommentTok{#}
\CommentTok{###################################################################################}
\CommentTok{# External Functions}
\CommentTok{###################################################################################}
\KeywordTok{library}\NormalTok{(igraph)}
\end{Highlighting}
\end{Shaded}

\begin{verbatim}
## 
## Attaching package: 'igraph'
\end{verbatim}

\begin{verbatim}
## The following objects are masked from 'package:stats':
## 
##     decompose, spectrum
\end{verbatim}

\begin{verbatim}
## The following object is masked from 'package:base':
## 
##     union
\end{verbatim}

\begin{Shaded}
\begin{Highlighting}[]
\KeywordTok{library}\NormalTok{(igraphdata)}
\CommentTok{###################################################################################}
\CommentTok{# Internal Functions}
\CommentTok{###################################################################################}
\NormalTok{getSubs<-}\ControlFlowTok{function}\NormalTok{(g,split) \{}
  \ControlFlowTok{if}\NormalTok{ (}\KeywordTok{class}\NormalTok{(g)}\OperatorTok{!=}\StringTok{"igraph"}\NormalTok{) }\KeywordTok{stop}\NormalTok{(}\StringTok{"Must input a graph"}\NormalTok{)}
  \ControlFlowTok{if}\NormalTok{ (}\KeywordTok{length}\NormalTok{(split)}\OperatorTok{!=}\KeywordTok{gorder}\NormalTok{(g)) }\KeywordTok{stop}\NormalTok{(}\StringTok{"split must have a value for each vertex"}\NormalTok{)}
\NormalTok{  split=}\KeywordTok{as.factor}\NormalTok{(split)}
\NormalTok{  subs=}\KeywordTok{list}\NormalTok{()}
  \ControlFlowTok{for}\NormalTok{ (i }\ControlFlowTok{in} \DecValTok{1}\OperatorTok{:}\KeywordTok{length}\NormalTok{(}\KeywordTok{levels}\NormalTok{(split))) \{}
\NormalTok{    subs[[i]]=}\KeywordTok{induced_subgraph}\NormalTok{(g,}\KeywordTok{which}\NormalTok{(}\KeywordTok{as.numeric}\NormalTok{(split)}\OperatorTok{==}\NormalTok{i))}
\NormalTok{  \}}
\NormalTok{  subs}
\NormalTok{\}}
\CommentTok{###################################################################################}
\CommentTok{# Save the environment }
\CommentTok{###################################################################################}
\NormalTok{parSave=}\KeywordTok{par}\NormalTok{(}\DataTypeTok{no.readonly =} \OtherTok{TRUE}\NormalTok{)}
\NormalTok{iSave=}\KeywordTok{igraph_options}\NormalTok{(}\DataTypeTok{annotate.plot=}\NormalTok{F,}\DataTypeTok{vertex.size=}\DecValTok{15}\NormalTok{,}\DataTypeTok{edge.arrow.size=}\NormalTok{.}\DecValTok{5}\NormalTok{)}
\CommentTok{#igraph_options(iSave)  #Back to the old values }
\CommentTok{#par(parSave)}
\CommentTok{###################################################################################}
\CommentTok{# Processing }
\CommentTok{###################################################################################}
\CommentTok{#1}
\KeywordTok{data}\NormalTok{(}\StringTok{"UKfaculty"}\NormalTok{)}
\CommentTok{#a}
\KeywordTok{summary}\NormalTok{(UKfaculty)}
\end{Highlighting}
\end{Shaded}

\begin{verbatim}
## IGRAPH 6f42903 D-W- 81 817 -- 
## + attr: Type (g/c), Date (g/c), Citation (g/c), Author (g/c), Group
## | (v/n), weight (e/n)
\end{verbatim}

\begin{Shaded}
\begin{Highlighting}[]
\CommentTok{#b}
\KeywordTok{V}\NormalTok{(UKfaculty)}\OperatorTok{$}\NormalTok{name=}\KeywordTok{V}\NormalTok{(UKfaculty)}
\CommentTok{#c}
\KeywordTok{igraph_options}\NormalTok{(}\DataTypeTok{vertex.size=}\DecValTok{5}\NormalTok{,}\DataTypeTok{edge.arrow.size=}\NormalTok{.}\DecValTok{3}\NormalTok{)}
\KeywordTok{set.seed}\NormalTok{(}\DecValTok{23}\NormalTok{);}\KeywordTok{plot}\NormalTok{(UKfaculty)}
\end{Highlighting}
\end{Shaded}

\includegraphics{HW6_files/figure-latex/unnamed-chunk-1-1.pdf}

\begin{Shaded}
\begin{Highlighting}[]
\CommentTok{#d}
\NormalTok{cols=}\KeywordTok{c}\NormalTok{(}\StringTok{"red"}\NormalTok{,}\StringTok{"green"}\NormalTok{,}\StringTok{"lightblue"}\NormalTok{,}\StringTok{"yellow"}\NormalTok{)}
\KeywordTok{V}\NormalTok{(UKfaculty)}\OperatorTok{$}\NormalTok{Group}
\end{Highlighting}
\end{Shaded}

\begin{verbatim}
##  [1] 3 1 3 3 2 2 2 1 3 2 1 2 2 1 1 2 3 1 1 1 1 2 2 1 1 1 2 2 1 2 1 1 2 1 1 3 1 3
## [39] 1 2 1 2 1 3 3 1 2 1 2 4 1 1 3 1 1 1 1 1 3 3 3 3 2 1 2 2 2 2 2 4 2 2 3 3 3 2
## [77] 2 3 1 1 3
\end{verbatim}

\begin{Shaded}
\begin{Highlighting}[]
\KeywordTok{plot}\NormalTok{(UKfaculty,}\DataTypeTok{layout=}\NormalTok{layout_with_kk,}\DataTypeTok{vertex.color=}\NormalTok{cols[}\KeywordTok{V}\NormalTok{(UKfaculty)}\OperatorTok{$}\NormalTok{Group],}
     \DataTypeTok{vertex.label=}\OtherTok{NA}\NormalTok{)}
\end{Highlighting}
\end{Shaded}

\includegraphics{HW6_files/figure-latex/unnamed-chunk-1-2.pdf}

\begin{Shaded}
\begin{Highlighting}[]
\KeywordTok{igraph_options}\NormalTok{(iSave)}
\CommentTok{#e}
\CommentTok{# desnity of the graph}
\KeywordTok{edge_density}\NormalTok{(UKfaculty)}
\end{Highlighting}
\end{Shaded}

\begin{verbatim}
## [1] 0.1260802
\end{verbatim}

\begin{Shaded}
\begin{Highlighting}[]
\CommentTok{#Alternatives}
\KeywordTok{mean}\NormalTok{(}\KeywordTok{degree}\NormalTok{(UKfaculty,}\DataTypeTok{mode=}\StringTok{"in"}\NormalTok{))}\OperatorTok{/}\NormalTok{(}\KeywordTok{gorder}\NormalTok{(UKfaculty)}\OperatorTok{-}\DecValTok{1}\NormalTok{)}
\end{Highlighting}
\end{Shaded}

\begin{verbatim}
## [1] 0.1260802
\end{verbatim}

\begin{Shaded}
\begin{Highlighting}[]
\KeywordTok{gsize}\NormalTok{(UKfaculty)}\OperatorTok{/}\NormalTok{(}\KeywordTok{gorder}\NormalTok{(UKfaculty)}\OperatorTok{*}\NormalTok{(}\KeywordTok{gorder}\NormalTok{(UKfaculty)}\OperatorTok{-}\DecValTok{1}\NormalTok{))}
\end{Highlighting}
\end{Shaded}

\begin{verbatim}
## [1] 0.1260802
\end{verbatim}

\begin{Shaded}
\begin{Highlighting}[]
\CommentTok{#f}
\NormalTok{UK <-}\StringTok{ }\KeywordTok{contract}\NormalTok{(UKfaculty, }\KeywordTok{V}\NormalTok{(UKfaculty)}\OperatorTok{$}\NormalTok{Group)}
\KeywordTok{summary}\NormalTok{(UK)}
\end{Highlighting}
\end{Shaded}

\begin{verbatim}
## IGRAPH 2b4a423 DNW- 4 817 -- 
## + attr: Type (g/c), Date (g/c), Citation (g/c), Author (g/c), name
## | (v/x), weight (e/n)
\end{verbatim}

\begin{Shaded}
\begin{Highlighting}[]
\KeywordTok{plot}\NormalTok{(UK,}\DataTypeTok{layout=}\NormalTok{layout_with_kk,}\DataTypeTok{vertex.color=}\NormalTok{cols)}
\end{Highlighting}
\end{Shaded}

\includegraphics{HW6_files/figure-latex/unnamed-chunk-1-3.pdf}

\begin{Shaded}
\begin{Highlighting}[]
\CommentTok{#g}
\KeywordTok{V}\NormalTok{(UK) }\CommentTok{# we have 4 groups which multiple vetices in them}
\end{Highlighting}
\end{Shaded}

\begin{verbatim}
## + 4/4 vertices, named, from 2b4a423:
## [1] c(2, 8, 11, 14, 15, 18, 19, 20, 21, 24, 25, 26, 29, 31, 32, 34, 35, 37, 39, 41, 43, 46, 48, 51, 52, 54, 55, 56, 57, 58, 64, 79, 80)
## [2] c(5, 6, 7, 10, 12, 13, 16, 22, 23, 27, 28, 30, 33, 40, 42, 47, 49, 63, 65, 66, 67, 68, 69, 71, 72, 76, 77)                         
## [3] c(1, 3, 4, 9, 17, 36, 38, 44, 45, 53, 59, 60, 61, 62, 73, 74, 75, 78, 81)                                                          
## [4] c(50, 70)
\end{verbatim}

\begin{Shaded}
\begin{Highlighting}[]
\KeywordTok{E}\NormalTok{(UK)}\OperatorTok{$}\NormalTok{weight =}\StringTok{ }\DecValTok{1} 
\NormalTok{UK =}\StringTok{ }\KeywordTok{simplify}\NormalTok{(UK)}
\CommentTok{#h}
\CommentTok{# h)    Rename the vertices with the group numbers and set their color}
\CommentTok{# attributes to the colors determined previously for the groups.}
\KeywordTok{V}\NormalTok{(UK) }\CommentTok{# we have 4 groups which multiple vetices in them}
\end{Highlighting}
\end{Shaded}

\begin{verbatim}
## + 4/4 vertices, named, from 2c4a25c:
## [1] c(2, 8, 11, 14, 15, 18, 19, 20, 21, 24, 25, 26, 29, 31, 32, 34, 35, 37, 39, 41, 43, 46, 48, 51, 52, 54, 55, 56, 57, 58, 64, 79, 80)
## [2] c(5, 6, 7, 10, 12, 13, 16, 22, 23, 27, 28, 30, 33, 40, 42, 47, 49, 63, 65, 66, 67, 68, 69, 71, 72, 76, 77)                         
## [3] c(1, 3, 4, 9, 17, 36, 38, 44, 45, 53, 59, 60, 61, 62, 73, 74, 75, 78, 81)                                                          
## [4] c(50, 70)
\end{verbatim}

\begin{Shaded}
\begin{Highlighting}[]
\CommentTok{# have we are grouping them as to just a number }
\KeywordTok{V}\NormalTok{(UK)}\OperatorTok{$}\NormalTok{name=}\DecValTok{1}\OperatorTok{:}\DecValTok{4}
\KeywordTok{V}\NormalTok{(UK)}
\end{Highlighting}
\end{Shaded}

\begin{verbatim}
## + 4/4 vertices, named, from 2c4a25c:
## [1] 1 2 3 4
\end{verbatim}

\begin{Shaded}
\begin{Highlighting}[]
\CommentTok{# giving them previously saved clors}
\KeywordTok{V}\NormalTok{(UK)}\OperatorTok{$}\NormalTok{color=cols}
\CommentTok{#i}
\KeywordTok{E}\NormalTok{(UK)}\OperatorTok{$}\NormalTok{width=}\KeywordTok{sqrt}\NormalTok{(}\KeywordTok{E}\NormalTok{(UK)}\OperatorTok{$}\NormalTok{weight)}
\CommentTok{#j}
\NormalTok{sz =}\StringTok{ }\KeywordTok{as.vector}\NormalTok{(}\KeywordTok{table}\NormalTok{(}\KeywordTok{V}\NormalTok{(UKfaculty)}\OperatorTok{$}\NormalTok{Group)) }\CommentTok{#Vertices in each group}
\KeywordTok{V}\NormalTok{(UK)}\OperatorTok{$}\NormalTok{size=}\DecValTok{5}\OperatorTok{*}\KeywordTok{sqrt}\NormalTok{(sz)}
\CommentTok{#k}
\KeywordTok{E}\NormalTok{(UK)}\OperatorTok{$}\NormalTok{color=}\StringTok{"black"}
\CommentTok{#l}
\KeywordTok{par}\NormalTok{(}\DataTypeTok{mfrow=}\KeywordTok{c}\NormalTok{(}\DecValTok{1}\NormalTok{,}\DecValTok{2}\NormalTok{))}
\KeywordTok{plot}\NormalTok{(UKfaculty,}\DataTypeTok{layout=}\NormalTok{layout_with_kk,}\DataTypeTok{vertex.color=}\NormalTok{cols[}\KeywordTok{V}\NormalTok{(UKfaculty)}\OperatorTok{$}\NormalTok{Group],}
     \DataTypeTok{vertex.label=}\OtherTok{NA}\NormalTok{,}\DataTypeTok{vertex.size=}\DecValTok{5}\NormalTok{,}\DataTypeTok{edge.arrow.size=}\NormalTok{.}\DecValTok{3}\NormalTok{)}
\KeywordTok{set.seed}\NormalTok{(}\DecValTok{23}\NormalTok{);}\KeywordTok{plot}\NormalTok{(UK)}
\end{Highlighting}
\end{Shaded}

\includegraphics{HW6_files/figure-latex/unnamed-chunk-1-4.pdf}

\begin{Shaded}
\begin{Highlighting}[]
\CommentTok{#m}
\KeywordTok{edge_density}\NormalTok{(UK)}
\end{Highlighting}
\end{Shaded}

\begin{verbatim}
## [1] 1
\end{verbatim}

\begin{Shaded}
\begin{Highlighting}[]
\CommentTok{#This is a complete graph}
\CommentTok{#n}
\CommentTok{#Group 1 has the most faculty members in the network, group 4 the least.}
\CommentTok{#The graph is complete, so at least one person in each group is friends }
\CommentTok{#with someone in each of the other groups. The number of friendships between}
\CommentTok{#group 1 and 2 are the greatest.There are few people in group 4, but they seem}
\CommentTok{#to have a significant number of friendships, especially with the people in group 1.}

\KeywordTok{igraph_options}\NormalTok{(iSave)}
\KeywordTok{par}\NormalTok{(parSave)}

\CommentTok{#2}
\CommentTok{#a}
\NormalTok{UK2=}\KeywordTok{as.undirected}\NormalTok{(UKfaculty)}
\KeywordTok{V}\NormalTok{(UK2)}\OperatorTok{$}\NormalTok{name=}\KeywordTok{V}\NormalTok{(UK2)}
\CommentTok{#b}
\KeywordTok{is.connected}\NormalTok{(UK2)}
\end{Highlighting}
\end{Shaded}

\begin{verbatim}
## [1] TRUE
\end{verbatim}

\begin{Shaded}
\begin{Highlighting}[]
\KeywordTok{edge_density}\NormalTok{(UK2)}
\end{Highlighting}
\end{Shaded}

\begin{verbatim}
## [1] 0.1780864
\end{verbatim}

\begin{Shaded}
\begin{Highlighting}[]
\CommentTok{#c}
\KeywordTok{vertex_connectivity}\NormalTok{(UK2)   }\CommentTok{#2}
\end{Highlighting}
\end{Shaded}

\begin{verbatim}
## [1] 2
\end{verbatim}

\begin{Shaded}
\begin{Highlighting}[]
\KeywordTok{edge_connectivity}\NormalTok{(UK2)     }\CommentTok{#2}
\end{Highlighting}
\end{Shaded}

\begin{verbatim}
## [1] 2
\end{verbatim}

\begin{Shaded}
\begin{Highlighting}[]
\KeywordTok{articulation_points}\NormalTok{(UK2) }\CommentTok{# None}
\end{Highlighting}
\end{Shaded}

\begin{verbatim}
## + 0/81 vertices, named, from 2cc2b2d:
\end{verbatim}

\begin{Shaded}
\begin{Highlighting}[]
\CommentTok{#No since the vertex connectivity is 2.}
\CommentTok{#d}
\KeywordTok{neighbors}\NormalTok{(UK2,}\StringTok{"1"}\NormalTok{)                 }\CommentTok{#No, 59 is not a neighbor}
\end{Highlighting}
\end{Shaded}

\begin{verbatim}
## + 9/81 vertices, named, from 2cc2b2d:
## [1] 4  36 38 44 45 52 61 62 81
\end{verbatim}

\begin{Shaded}
\begin{Highlighting}[]
\KeywordTok{vertex_connectivity}\NormalTok{(UK2,}\StringTok{"1"}\NormalTok{,}\StringTok{"59"}\NormalTok{)  }\CommentTok{#9 Vertex-independent paths}
\end{Highlighting}
\end{Shaded}

\begin{verbatim}
## [1] 9
\end{verbatim}

\begin{Shaded}
\begin{Highlighting}[]
\KeywordTok{shortest_paths}\NormalTok{(UK2,}\StringTok{"1"}\NormalTok{,}\StringTok{"59"}\NormalTok{)}
\end{Highlighting}
\end{Shaded}

\begin{verbatim}
## $vpath
## $vpath[[1]]
## + 3/81 vertices, named, from 2cc2b2d:
## [1] 1  38 59
## 
## 
## $epath
## NULL
## 
## $predecessors
## NULL
## 
## $inbound_edges
## NULL
\end{verbatim}

\begin{Shaded}
\begin{Highlighting}[]
\KeywordTok{edge_connectivity}\NormalTok{(UK2,}\StringTok{"1"}\NormalTok{,}\StringTok{"59"}\NormalTok{)    }\CommentTok{#9 Edge-independent paths}
\end{Highlighting}
\end{Shaded}

\begin{verbatim}
## [1] 9
\end{verbatim}

\begin{Shaded}
\begin{Highlighting}[]
\CommentTok{#e}
\NormalTok{cfg=}\KeywordTok{cluster_fast_greedy}\NormalTok{(UK2)}
\KeywordTok{length}\NormalTok{(cfg)      }\CommentTok{#5}
\end{Highlighting}
\end{Shaded}

\begin{verbatim}
## [1] 5
\end{verbatim}

\begin{Shaded}
\begin{Highlighting}[]
\CommentTok{#f}
\KeywordTok{plot_dendrogram}\NormalTok{(cfg)}
\end{Highlighting}
\end{Shaded}

\includegraphics{HW6_files/figure-latex/unnamed-chunk-1-5.pdf}

\begin{Shaded}
\begin{Highlighting}[]
\CommentTok{#g}
\NormalTok{subs=}\KeywordTok{getSubs}\NormalTok{(UK2,}\KeywordTok{membership}\NormalTok{(cfg))}
\KeywordTok{membership}\NormalTok{(cfg)[}\KeywordTok{c}\NormalTok{(}\DecValTok{1}\NormalTok{,}\DecValTok{59}\NormalTok{)]     }\CommentTok{#subgraph 3}
\end{Highlighting}
\end{Shaded}

\begin{verbatim}
##  1 59 
##  3  3
\end{verbatim}

\begin{Shaded}
\begin{Highlighting}[]
\KeywordTok{vertex_connectivity}\NormalTok{(subs[[}\DecValTok{3}\NormalTok{]],}\StringTok{"1"}\NormalTok{,}\StringTok{"59"}\NormalTok{)  }\CommentTok{#7}
\end{Highlighting}
\end{Shaded}

\begin{verbatim}
## [1] 7
\end{verbatim}

\begin{Shaded}
\begin{Highlighting}[]
\KeywordTok{shortest_paths}\NormalTok{(subs[[}\DecValTok{3}\NormalTok{]],}\StringTok{"1"}\NormalTok{,}\StringTok{"59"}\NormalTok{)       }\CommentTok{#Longer than before}
\end{Highlighting}
\end{Shaded}

\begin{verbatim}
## $vpath
## $vpath[[1]]
## + 4/18 vertices, named, from 2daac71:
## [1] 1  62 61 59
## 
## 
## $epath
## NULL
## 
## $predecessors
## NULL
## 
## $inbound_edges
## NULL
\end{verbatim}

\begin{Shaded}
\begin{Highlighting}[]
\KeywordTok{edge_connectivity}\NormalTok{(subs[[}\DecValTok{3}\NormalTok{]],}\StringTok{"1"}\NormalTok{,}\StringTok{"59"}\NormalTok{)    }\CommentTok{#7}
\end{Highlighting}
\end{Shaded}

\begin{verbatim}
## [1] 7
\end{verbatim}

\begin{Shaded}
\begin{Highlighting}[]
\KeywordTok{neighbors}\NormalTok{(subs[[}\DecValTok{3}\NormalTok{]],}\StringTok{"1"}\NormalTok{)}
\end{Highlighting}
\end{Shaded}

\begin{verbatim}
## + 7/18 vertices, named, from 2daac71:
## [1] 4  36 44 45 61 62 81
\end{verbatim}

\begin{Shaded}
\begin{Highlighting}[]
\KeywordTok{neighbors}\NormalTok{(UK2,}\StringTok{"1"}\NormalTok{)}
\end{Highlighting}
\end{Shaded}

\begin{verbatim}
## + 9/81 vertices, named, from 2cc2b2d:
## [1] 4  36 38 44 45 52 61 62 81
\end{verbatim}

\begin{Shaded}
\begin{Highlighting}[]
\CommentTok{#Vertex 1 lost 2 neighbors}
\CommentTok{#h}
\KeywordTok{lapply}\NormalTok{(subs,edge_density)}
\end{Highlighting}
\end{Shaded}

\begin{verbatim}
## [[1]]
## [1] 0.6374269
## 
## [[2]]
## [1] 0.6
## 
## [[3]]
## [1] 0.4052288
## 
## [[4]]
## [1] 0.4672365
## 
## [[5]]
## [1] 1
\end{verbatim}

\begin{Shaded}
\begin{Highlighting}[]
\CommentTok{#All of them are more dense}
\CommentTok{#i}
\KeywordTok{vertex_connectivity}\NormalTok{(subs[[}\DecValTok{3}\NormalTok{]])  }\CommentTok{#1, so there are cut vertices}
\end{Highlighting}
\end{Shaded}

\begin{verbatim}
## [1] 1
\end{verbatim}

\begin{Shaded}
\begin{Highlighting}[]
\KeywordTok{articulation_points}\NormalTok{(subs[[}\DecValTok{3}\NormalTok{]])  }\CommentTok{#Just one, number "81"}
\end{Highlighting}
\end{Shaded}

\begin{verbatim}
## + 1/18 vertex, named, from 2daac71:
## [1] 81
\end{verbatim}

\begin{Shaded}
\begin{Highlighting}[]
\NormalTok{UK3=subs[[}\DecValTok{3}\NormalTok{]]}\OperatorTok{-}\KeywordTok{vertex}\NormalTok{(}\StringTok{"81"}\NormalTok{)}
\NormalTok{UK3}
\end{Highlighting}
\end{Shaded}

\begin{verbatim}
## IGRAPH 2db3049 UNW- 17 57 -- 
## + attr: Type (g/c), Date (g/c), Citation (g/c), Author (g/c), Group
## | (v/n), name (v/n), weight (e/n)
## + edges from 2db3049 (vertex names):
##  [1]  1-- 4  3-- 4  3-- 9  4-- 9  3--17  4--17  9--17  1--36  3--36  4--36
## [11]  1--44 36--44  1--45  4--45  9--45 17--45 36--45 44--45  3--53  4--53
## [21]  9--53 17--53 36--53 44--53  3--59  9--59 17--59  4--60  9--60  1--61
## [31]  3--61  4--61  9--61 17--61 45--61 53--61 59--61  1--62  9--62 17--62
## [41] 44--62 61--62  4--74 36--74 59--74 61--74 62--74  4--75 17--75 53--75
## [51] 61--75 62--75 74--75  4--78 17--78 45--78 59--78
\end{verbatim}

\begin{Shaded}
\begin{Highlighting}[]
\KeywordTok{is.connected}\NormalTok{(UK3)}
\end{Highlighting}
\end{Shaded}

\begin{verbatim}
## [1] FALSE
\end{verbatim}

\begin{Shaded}
\begin{Highlighting}[]
\NormalTok{co=}\KeywordTok{components}\NormalTok{(UK3)}
\KeywordTok{set.seed}\NormalTok{(}\DecValTok{23}\NormalTok{);}\KeywordTok{plot}\NormalTok{(UK3,}\DataTypeTok{vertex.color=}\NormalTok{co}\OperatorTok{$}\NormalTok{membership)}
\CommentTok{#j}
\NormalTok{part=}\KeywordTok{cut_at}\NormalTok{(cfg,}\DecValTok{4}\NormalTok{)}
\CommentTok{#k}
\KeywordTok{table}\NormalTok{(}\DataTypeTok{sub=}\NormalTok{part, }\DataTypeTok{group=}\KeywordTok{V}\NormalTok{(UK2)}\OperatorTok{$}\NormalTok{Group)}
\end{Highlighting}
\end{Shaded}

\begin{verbatim}
##    group
## sub  1  2  3  4
##   1  8  0 19  2
##   2 19  0  0  0
##   3  0 27  0  0
##   4  6  0  0  0
\end{verbatim}

\begin{Shaded}
\begin{Highlighting}[]
\CommentTok{#All of group 2 was captured correctly in subgraph 3. Most of group 1 (19) was}
\CommentTok{#captured in subgraph 2, but 14 of them were captured in other subgraphs. All}
\CommentTok{#of group 3 and 4 were captured together in subgraph 1.}
\CommentTok{#l}
\NormalTok{cle=}\KeywordTok{cluster_leading_eigen}\NormalTok{(UK2)}
\KeywordTok{length}\NormalTok{(cle)     }\CommentTok{#5}
\end{Highlighting}
\end{Shaded}

\begin{verbatim}
## [1] 5
\end{verbatim}

\begin{Shaded}
\begin{Highlighting}[]
\CommentTok{#m}
\KeywordTok{table}\NormalTok{(}\DataTypeTok{sub=}\KeywordTok{membership}\NormalTok{(cle), }\DataTypeTok{group=}\KeywordTok{V}\NormalTok{(UK2)}\OperatorTok{$}\NormalTok{Group)}
\end{Highlighting}
\end{Shaded}

\begin{verbatim}
##    group
## sub  1  2  3  4
##   1  0  0 18  0
##   2 23  0  1  2
##   3  0 27  0  0
##   4  4  0  0  0
##   5  6  0  0  0
\end{verbatim}

\begin{Shaded}
\begin{Highlighting}[]
\CommentTok{#All of group 2 was captured correctly in subgraph 3. Most of group 1 (23) was}
\CommentTok{#captured in subgraph 2, but 10 of them ended up in other subgraphs. Every}
\CommentTok{#vertex of group 3 except for 1 was captured in subgraph 1.}
\CommentTok{#This one did a great job at recognizing group 2 and 3. Both recognized group 2.}
\CommentTok{#The spectral partitioning seems to have done a better job.}
\CommentTok{#n}
\KeywordTok{table}\NormalTok{(}\DataTypeTok{hierarchical=}\KeywordTok{membership}\NormalTok{(cfg), }\DataTypeTok{spectral=}\KeywordTok{membership}\NormalTok{(cle))}
\end{Highlighting}
\end{Shaded}

\begin{verbatim}
##             spectral
## hierarchical  1  2  3  4  5
##            1  0 19  0  0  0
##            2  1  6  0  4  0
##            3 17  1  0  0  0
##            4  0  0 27  0  0
##            5  0  0  0  0  6
\end{verbatim}

\begin{Shaded}
\begin{Highlighting}[]
\CommentTok{#Hierarchical 4 equals spectral 3. Hierarchical 5 equals spectral 5. The rest}
\CommentTok{#is different.}
\CommentTok{#o}
\NormalTok{lap <-}\StringTok{ }\KeywordTok{laplacian_matrix}\NormalTok{(UK2)}
\end{Highlighting}
\end{Shaded}

\includegraphics{HW6_files/figure-latex/unnamed-chunk-1-6.pdf}

\begin{Shaded}
\begin{Highlighting}[]
\NormalTok{eig=}\KeywordTok{eigen}\NormalTok{(lap)}
\KeywordTok{round}\NormalTok{(eig}\OperatorTok{$}\NormalTok{values,}\DecValTok{2}\NormalTok{)[}\KeywordTok{gorder}\NormalTok{(UK2)}\OperatorTok{-}\DecValTok{1}\NormalTok{]   }\CommentTok{#Fiedler value is 1.99}
\end{Highlighting}
\end{Shaded}

\begin{verbatim}
## [1] 1.99
\end{verbatim}

\begin{Shaded}
\begin{Highlighting}[]
\CommentTok{#p}
\NormalTok{Fied=eig}\OperatorTok{$}\NormalTok{vectors[,}\KeywordTok{gorder}\NormalTok{(UK2)}\OperatorTok{-}\DecValTok{1}\NormalTok{] }
\NormalTok{mem=Fied}\OperatorTok{<}\DecValTok{0}
\NormalTok{(}\DataTypeTok{subs=}\KeywordTok{getSubs}\NormalTok{(UK2,mem))}
\end{Highlighting}
\end{Shaded}

\begin{verbatim}
## [[1]]
## IGRAPH 2f1d6b5 UNW- 3 3 -- 
## + attr: Type (g/c), Date (g/c), Citation (g/c), Author (g/c), Group
## | (v/n), name (v/n), weight (e/n)
## + edges from 2f1d6b5 (vertex names):
## [1] 11--46 11--58 46--58
## 
## [[2]]
## IGRAPH 2f1d6b5 UNW- 78 542 -- 
## + attr: Type (g/c), Date (g/c), Citation (g/c), Author (g/c), Group
## | (v/n), name (v/n), weight (e/n)
## + edges from 2f1d6b5 (vertex names):
##  [1]  1-- 4  3-- 4  5-- 6  5-- 7  6-- 7  3-- 9  4-- 9  5-- 9  5--10  7--10
## [11]  5--12  7--12  5--13  7--13 10--13 12--13  2--15 14--15  5--16  7--16
## [21] 12--16 13--16  3--17  4--17  9--17  2--18  7--18 15--18 15--19 18--19
## [31]  2--20  6--20  8--20 14--20 15--20 18--20 19--20  2--21  8--21 10--21
## [41] 13--21 15--21 19--21  5--22  7--22 10--22 12--22 21--22  5--23  7--23
## [51] 10--23 12--23 13--23 22--23  2--25 15--25 20--25 21--25  2--26 14--26
## [61] 20--26 21--26  5--27  7--27 10--27 13--27 16--27 19--27 22--27 23--27
## + ... omitted several edges
\end{verbatim}

\begin{Shaded}
\begin{Highlighting}[]
\CommentTok{#q}
\KeywordTok{table}\NormalTok{(}\DataTypeTok{sub=}\NormalTok{mem, }\DataTypeTok{group=}\KeywordTok{V}\NormalTok{(UK2)}\OperatorTok{$}\NormalTok{Group)}
\end{Highlighting}
\end{Shaded}

\begin{verbatim}
##        group
## sub      1  2  3  4
##   FALSE  3  0  0  0
##   TRUE  30 27 19  2
\end{verbatim}

\begin{Shaded}
\begin{Highlighting}[]
\CommentTok{#A terrible job}

\CommentTok{#3}
\CommentTok{#Yeast data}
\KeywordTok{data}\NormalTok{(yeast)}

\CommentTok{#a}
\NormalTok{?yeast}
\end{Highlighting}
\end{Shaded}

\begin{verbatim}
## starting httpd help server ...
\end{verbatim}

\begin{verbatim}
##  done
\end{verbatim}

\begin{Shaded}
\begin{Highlighting}[]
\KeywordTok{summary}\NormalTok{(yeast)    }\CommentTok{#Very large network!}
\end{Highlighting}
\end{Shaded}

\begin{verbatim}
## IGRAPH 65c41bb UN-- 2617 11855 -- Yeast protein interactions, von Mering et al.
## + attr: name (g/c), Citation (g/c), Author (g/c), URL (g/c), Classes
## | (g/x), name (v/c), Class (v/c), Description (v/c), Confidence (e/c)
\end{verbatim}

\begin{Shaded}
\begin{Highlighting}[]
\CommentTok{#order is 2,617, size is 11,855}
\NormalTok{yeast}\OperatorTok{$}\NormalTok{URL}
\end{Highlighting}
\end{Shaded}

\begin{verbatim}
## [1] "http://www.nature.com/nature/journal/v417/n6887/full/nature750.html"
\end{verbatim}

\begin{Shaded}
\begin{Highlighting}[]
\CommentTok{#b}
\KeywordTok{table}\NormalTok{(}\KeywordTok{degree}\NormalTok{(yeast))}
\end{Highlighting}
\end{Shaded}

\begin{verbatim}
## 
##   1   2   3   4   5   6   7   8   9  10  11  12  13  14  15  16  17  18  19  20 
## 694 337 242 177 144 111  96  65  60  72  45  40  55  38  35  35  17  21  15  25 
##  21  22  23  24  25  26  27  28  29  30  31  32  33  34  35  36  37  38  39  40 
##  28  19  14  15   7  10   7   7   2   4  11  11   2  10   9   8   4   5   5  12 
##  41  42  43  44  45  46  47  48  49  50  51  52  53  54  55  56  57  58  60  62 
##   4   9   7   3   2   3   1   2   1   2   2   6   6   3   5   2   4   3   1   2 
##  68  73  86  90  91  93  94  95  97 101 102 104 106 107 108 113 114 115 118 
##   1   1   1   2   1   1   1   1   1   1   1  10   4   1   3   2   1   1   1
\end{verbatim}

\begin{Shaded}
\begin{Highlighting}[]
\KeywordTok{hist}\NormalTok{(}\KeywordTok{degree}\NormalTok{(yeast),}\DataTypeTok{col=}\StringTok{"turquoise"}\NormalTok{)}
\end{Highlighting}
\end{Shaded}

\includegraphics{HW6_files/figure-latex/unnamed-chunk-1-7.pdf}

\begin{Shaded}
\begin{Highlighting}[]
\CommentTok{#Most proteins interact with only a few other proteins, but some of them}
\CommentTok{#interact with many.}

\CommentTok{#c}
\CommentTok{#Components}
\KeywordTok{is_connected}\NormalTok{(yeast)}
\end{Highlighting}
\end{Shaded}

\begin{verbatim}
## [1] FALSE
\end{verbatim}

\begin{Shaded}
\begin{Highlighting}[]
\NormalTok{co=}\KeywordTok{components}\NormalTok{(yeast)}
\NormalTok{co}\OperatorTok{$}\NormalTok{no     }\CommentTok{#92 components}
\end{Highlighting}
\end{Shaded}

\begin{verbatim}
## [1] 92
\end{verbatim}

\begin{Shaded}
\begin{Highlighting}[]
\NormalTok{co}\OperatorTok{$}\NormalTok{csize}
\end{Highlighting}
\end{Shaded}

\begin{verbatim}
##  [1] 2375    3    5    5    7    6    4    3    5    2    7    2    2    7    3
## [16]    2    2    4    2    2    2    2    2    2    2    2    2    2    2    2
## [31]    2    4    2    2    2    2    2    5    2    2    3    2    2    3    2
## [46]    2    5    3    3    2    2    2    2    3    3    2    2    2    2    2
## [61]    2    2    2    2    2    2    2    2    2    2    2    3    5    3    3
## [76]    4    2    2    2    2    2    3    2    2    4    2    2    2    2    2
## [91]    2    2
\end{verbatim}

\begin{Shaded}
\begin{Highlighting}[]
\CommentTok{#The first one is a giant component, consisting of 2375 proteins.}

\CommentTok{#Break graph into component graphs}
\NormalTok{comps=}\KeywordTok{decompose}\NormalTok{(yeast)}
\KeywordTok{table}\NormalTok{(}\KeywordTok{sapply}\NormalTok{(comps,gorder))}
\end{Highlighting}
\end{Shaded}

\begin{verbatim}
## 
##    2    3    4    5    6    7 2375 
##   63   13    5    6    1    3    1
\end{verbatim}

\begin{Shaded}
\begin{Highlighting}[]
\CommentTok{#component with 2375 vertices is the giant component.}
\NormalTok{giant=comps[[}\DecValTok{1}\NormalTok{]]}

\KeywordTok{summary}\NormalTok{(giant)}
\end{Highlighting}
\end{Shaded}

\begin{verbatim}
## IGRAPH 2f7685e UN-- 2375 11693 -- Yeast protein interactions, von Mering et al.
## + attr: name (g/c), Citation (g/c), Author (g/c), URL (g/c), Classes
## | (g/x), name (v/c), Class (v/c), Description (v/c), Confidence (e/c)
\end{verbatim}

\begin{Shaded}
\begin{Highlighting}[]
\KeywordTok{gorder}\NormalTok{(giant)}\OperatorTok{*}\DecValTok{100}\OperatorTok{/}\KeywordTok{gorder}\NormalTok{(yeast)}
\end{Highlighting}
\end{Shaded}

\begin{verbatim}
## [1] 90.75277
\end{verbatim}

\begin{Shaded}
\begin{Highlighting}[]
\CommentTok{#Giant contains more than 90% of the vertices}

\CommentTok{#d}
\KeywordTok{mean_distance}\NormalTok{(giant)   }\CommentTok{#Average shortest path over all pairs of vertices}
\end{Highlighting}
\end{Shaded}

\begin{verbatim}
## [1] 5.09597
\end{verbatim}

\begin{Shaded}
\begin{Highlighting}[]
\KeywordTok{diameter}\NormalTok{(giant)        }\CommentTok{#Longest shortest path over all pairs of vertices}
\end{Highlighting}
\end{Shaded}

\begin{verbatim}
## [1] 15
\end{verbatim}

\begin{Shaded}
\begin{Highlighting}[]
\KeywordTok{diameter}\NormalTok{(giant)}\OperatorTok{/}\KeywordTok{gorder}\NormalTok{(giant)}\OperatorTok{*}\DecValTok{100}
\end{Highlighting}
\end{Shaded}

\begin{verbatim}
## [1] 0.6315789
\end{verbatim}

\begin{Shaded}
\begin{Highlighting}[]
\CommentTok{#On average, we can get from one protein to any other protein by following}
\CommentTok{#about 5 connections. Worst case scenario, if we pick the 2 worst proteins,}
\CommentTok{#it will take following 15 connections. Even the worst case scenario only}
\CommentTok{#passes through .6% of the proteins in the network.}

\CommentTok{#e}
\KeywordTok{vertex_connectivity}\NormalTok{(giant)}
\end{Highlighting}
\end{Shaded}

\begin{verbatim}
## [1] 1
\end{verbatim}

\begin{Shaded}
\begin{Highlighting}[]
\KeywordTok{edge_connectivity}\NormalTok{(giant)}
\end{Highlighting}
\end{Shaded}

\begin{verbatim}
## [1] 1
\end{verbatim}

\begin{Shaded}
\begin{Highlighting}[]
\NormalTok{cutVertices=}\KeywordTok{articulation_points}\NormalTok{(giant) }\CommentTok{#Find all cut vertices}
\KeywordTok{length}\NormalTok{(cutVertices)}
\end{Highlighting}
\end{Shaded}

\begin{verbatim}
## [1] 350
\end{verbatim}

\begin{Shaded}
\begin{Highlighting}[]
\KeywordTok{length}\NormalTok{(cutVertices)}\OperatorTok{/}\KeywordTok{gorder}\NormalTok{(giant)}
\end{Highlighting}
\end{Shaded}

\begin{verbatim}
## [1] 0.1473684
\end{verbatim}

\begin{Shaded}
\begin{Highlighting}[]
\CommentTok{#Almost 15% of vertices will disconnect the graph if removed}

\CommentTok{#f}
\CommentTok{#Frequency table}
\KeywordTok{table}\NormalTok{(}\KeywordTok{V}\NormalTok{(giant)}\OperatorTok{$}\NormalTok{Class)}
\end{Highlighting}
\end{Shaded}

\begin{verbatim}
## 
##   A   B   C   D   E   F   G   M   O   P   R   T   U 
##  51  98 122 238  95 171  96 278 171 248  45 240 483
\end{verbatim}

\begin{Shaded}
\begin{Highlighting}[]
\KeywordTok{table}\NormalTok{(}\KeywordTok{vertex_attr}\NormalTok{(giant,}\StringTok{"Class"}\NormalTok{)) }\CommentTok{#Alternative method}
\end{Highlighting}
\end{Shaded}

\begin{verbatim}
## 
##   A   B   C   D   E   F   G   M   O   P   R   T   U 
##  51  98 122 238  95 171  96 278 171 248  45 240 483
\end{verbatim}

\begin{Shaded}
\begin{Highlighting}[]
\CommentTok{#g}
\NormalTok{cfg=}\KeywordTok{cluster_fast_greedy}\NormalTok{(giant)}
\KeywordTok{length}\NormalTok{(cfg)             }\CommentTok{#31 subgraphs}
\end{Highlighting}
\end{Shaded}

\begin{verbatim}
## [1] 31
\end{verbatim}

\begin{Shaded}
\begin{Highlighting}[]
\KeywordTok{sizes}\NormalTok{(cfg)}
\end{Highlighting}
\end{Shaded}

\begin{verbatim}
## Community sizes
##   1   2   3   4   5   6   7   8   9  10  11  12  13  14  15  16  17  18  19  20 
## 182  98 255 733 148  82 186 363  55  25  30  22  34  24  15  15  11  15   6  10 
##  21  22  23  24  25  26  27  28  29  30  31 
##   8   9   4   5   6   7   8   7   5   4   3
\end{verbatim}

\begin{Shaded}
\begin{Highlighting}[]
\NormalTok{mem=}\KeywordTok{membership}\NormalTok{(cfg)}

\CommentTok{#h}
\CommentTok{#Contingency table}
\KeywordTok{table}\NormalTok{(mem, }\KeywordTok{V}\NormalTok{(giant)}\OperatorTok{$}\NormalTok{Class)}
\end{Highlighting}
\end{Shaded}

\begin{verbatim}
##     
## mem    A   B   C   D   E   F   G   M   O   P   R   T   U
##   1    0   0   0   1   3   7   0   6   3 110   2  35  14
##   2    0   2   2   7   1   1   1   4  39   5   0   4  27
##   3    1   9   7  18   4   8   4  20  10  23   8  74  64
##   4   25  11  10  22  72  84  81 168  14  75  16  27 121
##   5    1   7   5  14   0   4   0   2   3   6   1  34  68
##   6    1  24   1   4   1   4   0   7   0   1   0  19  16
##   7    6  18   6  76   7   9   3   7   8   5   1   7  33
##   8    8  12  67  59   1  34   0  19  60  10   7   6  73
##   9    4   1   7   7   2  10   5   3   2   0   3   0  11
##   10   0   0   0   6   0   0   0   2   0   5   0  11   1
##   11   0   9   0  10   1   3   0   0   0   0   0   2   4
##   12   0   1   3   0   0   0   0   6  10   0   0   0   2
##   13   0   1   1   2   0   1   0   0   2   0   0  16  10
##   14   1   0   4   1   0   1   0   0   4   0   1   0  11
##   15   0   1   0   0   0   2   0   2   0   0   1   0   8
##   16   0   1   2   0   0   1   0   0  10   0   0   0   0
##   17   0   0   1   3   0   0   0   2   0   0   0   2   3
##   18   0   0   0   0   3   1   0   9   0   0   1   0   1
##   19   0   1   1   1   0   0   0   0   0   0   0   0   3
##   20   0   0   0   6   0   0   0   1   0   0   0   1   2
##   21   1   0   0   0   0   0   0   0   6   0   0   1   0
##   22   0   0   0   0   0   0   0   1   0   0   0   0   8
##   23   0   0   0   0   0   0   0   4   0   0   0   0   0
##   24   0   0   0   0   0   0   2   2   0   0   0   1   0
##   25   0   0   0   0   0   0   0   5   0   0   0   0   0
##   26   0   0   1   0   0   0   0   4   0   0   1   0   1
##   27   3   0   4   0   0   1   0   0   0   0   0   0   0
##   28   0   0   0   0   0   0   0   0   0   6   0   0   0
##   29   0   0   0   1   0   0   0   1   0   0   3   0   0
##   30   0   0   0   0   0   0   0   0   0   2   0   0   2
##   31   0   0   0   0   0   0   0   3   0   0   0   0   0
\end{verbatim}

\begin{Shaded}
\begin{Highlighting}[]
\CommentTok{#Members of the first subgraph mostly belong to class "P"}
\CommentTok{#Members of subgraph 4 are spread over all classes}

\CommentTok{#i}
\NormalTok{cle=}\KeywordTok{cluster_leading_eigen}\NormalTok{(giant)}
\KeywordTok{length}\NormalTok{(cle)             }\CommentTok{#8 subgraphs}
\end{Highlighting}
\end{Shaded}

\begin{verbatim}
## [1] 8
\end{verbatim}

\begin{Shaded}
\begin{Highlighting}[]
\KeywordTok{sizes}\NormalTok{(cle)               }
\end{Highlighting}
\end{Shaded}

\begin{verbatim}
## Community sizes
##    1    2    3    4    5    6    7    8 
##  179  175  159   78 1484  177  104   19
\end{verbatim}

\begin{Shaded}
\begin{Highlighting}[]
\CommentTok{#j}
\NormalTok{(}\DataTypeTok{tab=}\KeywordTok{table}\NormalTok{(}\KeywordTok{membership}\NormalTok{(cle), }\KeywordTok{V}\NormalTok{(giant)}\OperatorTok{$}\NormalTok{Class))}
\end{Highlighting}
\end{Shaded}

\begin{verbatim}
##    
##       A   B   C   D   E   F   G   M   O   P   R   T   U
##   1   1   8   5  14   0   7   2   3   5  23   2  41  64
##   2   0   0   0   1   4   7   1   7   3 105   1  32  13
##   3   1  10   3  13   4   7   0   8   9   3   2  57  35
##   4   1  28   0   2   5   2   0   1   0   2   0  19  18
##   5  37  46 100 183  65 104  74 225 133  78  36  70 310
##   6   3   5  10  15  12  37  16  18  12  20   1   5  21
##   7   8   0   4  10   5   7   3  16   2  17   3  15  14
##   8   0   1   0   0   0   0   0   0   7   0   0   1   8
\end{verbatim}

\begin{Shaded}
\begin{Highlighting}[]
\CommentTok{#Members of subgraph 2 mostly belong to class "P"}
\CommentTok{#Members of subgraph 5 are spread over all classes}
\end{Highlighting}
\end{Shaded}

\end{document}
