% Options for packages loaded elsewhere
\PassOptionsToPackage{unicode}{hyperref}
\PassOptionsToPackage{hyphens}{url}
%
\documentclass[
]{article}
\usepackage{lmodern}
\usepackage{amssymb,amsmath}
\usepackage{ifxetex,ifluatex}
\ifnum 0\ifxetex 1\fi\ifluatex 1\fi=0 % if pdftex
  \usepackage[T1]{fontenc}
  \usepackage[utf8]{inputenc}
  \usepackage{textcomp} % provide euro and other symbols
\else % if luatex or xetex
  \usepackage{unicode-math}
  \defaultfontfeatures{Scale=MatchLowercase}
  \defaultfontfeatures[\rmfamily]{Ligatures=TeX,Scale=1}
\fi
% Use upquote if available, for straight quotes in verbatim environments
\IfFileExists{upquote.sty}{\usepackage{upquote}}{}
\IfFileExists{microtype.sty}{% use microtype if available
  \usepackage[]{microtype}
  \UseMicrotypeSet[protrusion]{basicmath} % disable protrusion for tt fonts
}{}
\makeatletter
\@ifundefined{KOMAClassName}{% if non-KOMA class
  \IfFileExists{parskip.sty}{%
    \usepackage{parskip}
  }{% else
    \setlength{\parindent}{0pt}
    \setlength{\parskip}{6pt plus 2pt minus 1pt}}
}{% if KOMA class
  \KOMAoptions{parskip=half}}
\makeatother
\usepackage{xcolor}
\IfFileExists{xurl.sty}{\usepackage{xurl}}{} % add URL line breaks if available
\IfFileExists{bookmark.sty}{\usepackage{bookmark}}{\usepackage{hyperref}}
\hypersetup{
  pdftitle={HW4-ES.R},
  pdfauthor={Malik},
  hidelinks,
  pdfcreator={LaTeX via pandoc}}
\urlstyle{same} % disable monospaced font for URLs
\usepackage[margin=1in]{geometry}
\usepackage{color}
\usepackage{fancyvrb}
\newcommand{\VerbBar}{|}
\newcommand{\VERB}{\Verb[commandchars=\\\{\}]}
\DefineVerbatimEnvironment{Highlighting}{Verbatim}{commandchars=\\\{\}}
% Add ',fontsize=\small' for more characters per line
\usepackage{framed}
\definecolor{shadecolor}{RGB}{248,248,248}
\newenvironment{Shaded}{\begin{snugshade}}{\end{snugshade}}
\newcommand{\AlertTok}[1]{\textcolor[rgb]{0.94,0.16,0.16}{#1}}
\newcommand{\AnnotationTok}[1]{\textcolor[rgb]{0.56,0.35,0.01}{\textbf{\textit{#1}}}}
\newcommand{\AttributeTok}[1]{\textcolor[rgb]{0.77,0.63,0.00}{#1}}
\newcommand{\BaseNTok}[1]{\textcolor[rgb]{0.00,0.00,0.81}{#1}}
\newcommand{\BuiltInTok}[1]{#1}
\newcommand{\CharTok}[1]{\textcolor[rgb]{0.31,0.60,0.02}{#1}}
\newcommand{\CommentTok}[1]{\textcolor[rgb]{0.56,0.35,0.01}{\textit{#1}}}
\newcommand{\CommentVarTok}[1]{\textcolor[rgb]{0.56,0.35,0.01}{\textbf{\textit{#1}}}}
\newcommand{\ConstantTok}[1]{\textcolor[rgb]{0.00,0.00,0.00}{#1}}
\newcommand{\ControlFlowTok}[1]{\textcolor[rgb]{0.13,0.29,0.53}{\textbf{#1}}}
\newcommand{\DataTypeTok}[1]{\textcolor[rgb]{0.13,0.29,0.53}{#1}}
\newcommand{\DecValTok}[1]{\textcolor[rgb]{0.00,0.00,0.81}{#1}}
\newcommand{\DocumentationTok}[1]{\textcolor[rgb]{0.56,0.35,0.01}{\textbf{\textit{#1}}}}
\newcommand{\ErrorTok}[1]{\textcolor[rgb]{0.64,0.00,0.00}{\textbf{#1}}}
\newcommand{\ExtensionTok}[1]{#1}
\newcommand{\FloatTok}[1]{\textcolor[rgb]{0.00,0.00,0.81}{#1}}
\newcommand{\FunctionTok}[1]{\textcolor[rgb]{0.00,0.00,0.00}{#1}}
\newcommand{\ImportTok}[1]{#1}
\newcommand{\InformationTok}[1]{\textcolor[rgb]{0.56,0.35,0.01}{\textbf{\textit{#1}}}}
\newcommand{\KeywordTok}[1]{\textcolor[rgb]{0.13,0.29,0.53}{\textbf{#1}}}
\newcommand{\NormalTok}[1]{#1}
\newcommand{\OperatorTok}[1]{\textcolor[rgb]{0.81,0.36,0.00}{\textbf{#1}}}
\newcommand{\OtherTok}[1]{\textcolor[rgb]{0.56,0.35,0.01}{#1}}
\newcommand{\PreprocessorTok}[1]{\textcolor[rgb]{0.56,0.35,0.01}{\textit{#1}}}
\newcommand{\RegionMarkerTok}[1]{#1}
\newcommand{\SpecialCharTok}[1]{\textcolor[rgb]{0.00,0.00,0.00}{#1}}
\newcommand{\SpecialStringTok}[1]{\textcolor[rgb]{0.31,0.60,0.02}{#1}}
\newcommand{\StringTok}[1]{\textcolor[rgb]{0.31,0.60,0.02}{#1}}
\newcommand{\VariableTok}[1]{\textcolor[rgb]{0.00,0.00,0.00}{#1}}
\newcommand{\VerbatimStringTok}[1]{\textcolor[rgb]{0.31,0.60,0.02}{#1}}
\newcommand{\WarningTok}[1]{\textcolor[rgb]{0.56,0.35,0.01}{\textbf{\textit{#1}}}}
\usepackage{graphicx,grffile}
\makeatletter
\def\maxwidth{\ifdim\Gin@nat@width>\linewidth\linewidth\else\Gin@nat@width\fi}
\def\maxheight{\ifdim\Gin@nat@height>\textheight\textheight\else\Gin@nat@height\fi}
\makeatother
% Scale images if necessary, so that they will not overflow the page
% margins by default, and it is still possible to overwrite the defaults
% using explicit options in \includegraphics[width, height, ...]{}
\setkeys{Gin}{width=\maxwidth,height=\maxheight,keepaspectratio}
% Set default figure placement to htbp
\makeatletter
\def\fps@figure{htbp}
\makeatother
\setlength{\emergencystretch}{3em} % prevent overfull lines
\providecommand{\tightlist}{%
  \setlength{\itemsep}{0pt}\setlength{\parskip}{0pt}}
\setcounter{secnumdepth}{-\maxdimen} % remove section numbering

\title{HW4-ES.R}
\author{Malik}
\date{2020-10-05}

\begin{document}
\maketitle

\begin{Shaded}
\begin{Highlighting}[]
\CommentTok{###################################################################################}
\CommentTok{#}
\CommentTok{# HW4.R}
\CommentTok{#}
\CommentTok{###################################################################################}
\CommentTok{# External Functions}
\CommentTok{###################################################################################}
\KeywordTok{library}\NormalTok{(igraph)}
\end{Highlighting}
\end{Shaded}

\begin{verbatim}
## 
## Attaching package: 'igraph'
\end{verbatim}

\begin{verbatim}
## The following objects are masked from 'package:stats':
## 
##     decompose, spectrum
\end{verbatim}

\begin{verbatim}
## The following object is masked from 'package:base':
## 
##     union
\end{verbatim}

\begin{Shaded}
\begin{Highlighting}[]
\KeywordTok{library}\NormalTok{(igraphdata)}
\CommentTok{###################################################################################}
\CommentTok{# Internal Functions}
\CommentTok{###################################################################################}
\CommentTok{###################################################################################}
\CommentTok{# Save the environment }
\CommentTok{###################################################################################}
\NormalTok{parSave=}\KeywordTok{par}\NormalTok{(}\DataTypeTok{no.readonly =} \OtherTok{TRUE}\NormalTok{)}
\NormalTok{iSave=}\KeywordTok{igraph_options}\NormalTok{(}\DataTypeTok{annotate.plot=}\NormalTok{F,}\DataTypeTok{vertex.size=}\DecValTok{15}\NormalTok{,}\DataTypeTok{edge.arrow.size=}\NormalTok{.}\DecValTok{5}\NormalTok{)}
\KeywordTok{options}\NormalTok{(}\DataTypeTok{stringsAsFactors =}\NormalTok{ F)}
\CommentTok{#igraph_options(iSave)  #Back to the old values }
\CommentTok{#par(parSave)}
\CommentTok{###################################################################################}
\CommentTok{# Processing }
\CommentTok{###################################################################################}
\CommentTok{#1}
\NormalTok{tr=}\KeywordTok{graph_from_literal}\NormalTok{(A}\OperatorTok{-+}\NormalTok{X}\OperatorTok{:}\NormalTok{R,B}\OperatorTok{-+}\NormalTok{A}\OperatorTok{:}\NormalTok{C,D}\OperatorTok{-+}\NormalTok{O}\OperatorTok{:}\NormalTok{K}\OperatorTok{:}\NormalTok{M,E}\OperatorTok{-+}\NormalTok{B}\OperatorTok{:}\NormalTok{Q}\OperatorTok{:}\NormalTok{G}\OperatorTok{:}\NormalTok{H,H}\OperatorTok{-+}\NormalTok{F,}
\NormalTok{                     J}\OperatorTok{-+}\NormalTok{T,O}\OperatorTok{-+}\NormalTok{N,P}\OperatorTok{-+}\NormalTok{W,Q}\OperatorTok{-+}\NormalTok{D}\OperatorTok{:}\NormalTok{I}\OperatorTok{:}\NormalTok{P}\OperatorTok{:}\NormalTok{U,T}\OperatorTok{-+}\NormalTok{L}\OperatorTok{:}\NormalTok{S}\OperatorTok{:}\NormalTok{Z,U}\OperatorTok{-+}\NormalTok{J,W}\OperatorTok{-+}\NormalTok{Y)}
\NormalTok{tr}
\end{Highlighting}
\end{Shaded}

\begin{verbatim}
## IGRAPH ea152a3 DN-- 25 24 -- 
## + attr: name (v/c)
## + edges from ea152a3 (vertex names):
##  [1] A->X A->R B->A B->C D->O D->K D->M O->N E->B E->Q E->G E->H Q->D Q->P Q->I
## [16] Q->U H->F J->T T->L T->S T->Z P->W W->Y U->J
\end{verbatim}

\begin{Shaded}
\begin{Highlighting}[]
\KeywordTok{igraph_options}\NormalTok{(}\DataTypeTok{vertex.size=}\DecValTok{10}\NormalTok{,}\DataTypeTok{edge.arrow.size=}\NormalTok{.}\DecValTok{5}\NormalTok{)}
\KeywordTok{set.seed}\NormalTok{(}\DecValTok{11}\NormalTok{);}\KeywordTok{plot}\NormalTok{(tr)}
\end{Highlighting}
\end{Shaded}

\includegraphics{HW4-ES_files/figure-latex/unnamed-chunk-1-1.pdf}

\begin{Shaded}
\begin{Highlighting}[]
\CommentTok{#a}
\CommentTok{#This is a rooted tree, and the root is E.}
\CommentTok{#b}
\KeywordTok{plot}\NormalTok{(tr,}\DataTypeTok{layout=}\NormalTok{layout_as_tree)}
\end{Highlighting}
\end{Shaded}

\includegraphics{HW4-ES_files/figure-latex/unnamed-chunk-1-2.pdf}

\begin{Shaded}
\begin{Highlighting}[]
\CommentTok{#c}
\CommentTok{#Yes, Y is a descendant of Q}
\CommentTok{#d}
\CommentTok{#A and C are the children of B}
\CommentTok{#e}
\CommentTok{#The leaves are X,R,C,N,K,M,Y,I,L,S,Z,G,F}
\CommentTok{#f}
\KeywordTok{gorder}\NormalTok{(tr);}\KeywordTok{gsize}\NormalTok{(tr)}
\end{Highlighting}
\end{Shaded}

\begin{verbatim}
## [1] 25
\end{verbatim}

\begin{verbatim}
## [1] 24
\end{verbatim}

\begin{Shaded}
\begin{Highlighting}[]
\CommentTok{#This tells us that the graph is a tree (m=n-1)}
\CommentTok{#g}
\KeywordTok{table}\NormalTok{(}\KeywordTok{degree}\NormalTok{(tr,}\DataTypeTok{mode=}\StringTok{"in"}\NormalTok{))}
\end{Highlighting}
\end{Shaded}

\begin{verbatim}
## 
##  0  1 
##  1 24
\end{verbatim}

\begin{Shaded}
\begin{Highlighting}[]
\CommentTok{#Almost all vertices have one edge going into it, except one (the root)}
\KeywordTok{gsize}\NormalTok{(tr)}\OperatorTok{/}\KeywordTok{gorder}\NormalTok{(tr) }\CommentTok{# that means}
\end{Highlighting}
\end{Shaded}

\begin{verbatim}
## [1] 0.96
\end{verbatim}

\begin{Shaded}
\begin{Highlighting}[]
\CommentTok{#h}
\KeywordTok{table}\NormalTok{(}\KeywordTok{degree}\NormalTok{(tr,}\DataTypeTok{mode=}\StringTok{"out"}\NormalTok{))}
\end{Highlighting}
\end{Shaded}

\begin{verbatim}
## 
##  0  1  2  3  4 
## 13  6  2  2  2
\end{verbatim}

\begin{Shaded}
\begin{Highlighting}[]
\KeywordTok{hist}\NormalTok{(}\KeywordTok{degree}\NormalTok{(tr,}\DataTypeTok{mode=}\StringTok{"out"}\NormalTok{))}
\end{Highlighting}
\end{Shaded}

\includegraphics{HW4-ES_files/figure-latex/unnamed-chunk-1-3.pdf}

\begin{Shaded}
\begin{Highlighting}[]
\KeywordTok{plot}\NormalTok{(}\DecValTok{0}\OperatorTok{:}\KeywordTok{max}\NormalTok{(}\KeywordTok{degree}\NormalTok{(tr,}\DataTypeTok{mode=}\StringTok{"out"}\NormalTok{)),}\KeywordTok{degree_distribution}\NormalTok{(tr,}\DataTypeTok{mode=}\StringTok{"out"}\NormalTok{),}
     \DataTypeTok{pch=}\DecValTok{20}\NormalTok{,}\DataTypeTok{xlab=}\StringTok{"degree"}\NormalTok{,}\DataTypeTok{ylab=}\OtherTok{NA}\NormalTok{,}\DataTypeTok{type=}\StringTok{'h'}\NormalTok{,}\DataTypeTok{ylim=}\KeywordTok{c}\NormalTok{(}\DecValTok{0}\NormalTok{,.}\DecValTok{6}\NormalTok{))           }
\end{Highlighting}
\end{Shaded}

\includegraphics{HW4-ES_files/figure-latex/unnamed-chunk-1-4.pdf}

\begin{Shaded}
\begin{Highlighting}[]
\CommentTok{#13 vertices have no outgoing edges; these are the leaves.}
\CommentTok{#The mean out-degree equals the mean in-degree, 0.96}
\CommentTok{#i}
\KeywordTok{sort}\NormalTok{(}\KeywordTok{degree}\NormalTok{(tr),}\DataTypeTok{decreasing=}\NormalTok{T)     }\CommentTok{#Q has the highest degree}
\end{Highlighting}
\end{Shaded}

\begin{verbatim}
## Q D E T A B O H J P W U X R C K M G F N I L S Z Y 
## 5 4 4 4 3 3 2 2 2 2 2 2 1 1 1 1 1 1 1 1 1 1 1 1 1
\end{verbatim}

\begin{Shaded}
\begin{Highlighting}[]
\CommentTok{#j}
\KeywordTok{sort}\NormalTok{(}\KeywordTok{knn}\NormalTok{(tr)}\OperatorTok{$}\NormalTok{knn, }\DataTypeTok{decreasing=}\NormalTok{T)   }\CommentTok{#I has avg. neighbor degree of 5}
\end{Highlighting}
\end{Shaded}

\begin{verbatim}
##        I        K        M        G        L        S        Z        P 
## 5.000000 4.000000 4.000000 4.000000 4.000000 4.000000 4.000000 3.500000 
##        U        X        R        C        J        E        B        Q 
## 3.500000 3.000000 3.000000 3.000000 3.000000 2.750000 2.666667 2.600000 
##        O        H        D        F        N        Y        A        W 
## 2.500000 2.500000 2.250000 2.000000 2.000000 2.000000 1.666667 1.500000 
##        T 
## 1.250000
\end{verbatim}

\begin{Shaded}
\begin{Highlighting}[]
\KeywordTok{neighbors}\NormalTok{(tr,}\StringTok{"I"}\NormalTok{,}\DataTypeTok{mode=}\StringTok{"all"}\NormalTok{)      }\CommentTok{#Just Q}
\end{Highlighting}
\end{Shaded}

\begin{verbatim}
## + 1/25 vertex, named, from ea152a3:
## [1] Q
\end{verbatim}

\begin{Shaded}
\begin{Highlighting}[]
\CommentTok{#k}
\NormalTok{isg=}\KeywordTok{induced_subgraph}\NormalTok{(tr,}\KeywordTok{c}\NormalTok{(}\StringTok{"D"}\NormalTok{,}\StringTok{"I"}\NormalTok{,}\StringTok{"P"}\NormalTok{,}\StringTok{"Q"}\NormalTok{,}\StringTok{"U"}\NormalTok{))}
\KeywordTok{set.seed}\NormalTok{(}\DecValTok{23}\NormalTok{);}\KeywordTok{plot}\NormalTok{(isg)}
\end{Highlighting}
\end{Shaded}

\includegraphics{HW4-ES_files/figure-latex/unnamed-chunk-1-5.pdf}

\begin{Shaded}
\begin{Highlighting}[]
\KeywordTok{plot}\NormalTok{(isg,}\DataTypeTok{layout=}\NormalTok{layout_as_star) }\CommentTok{#This is not working here}
\end{Highlighting}
\end{Shaded}

\includegraphics{HW4-ES_files/figure-latex/unnamed-chunk-1-6.pdf}

\begin{Shaded}
\begin{Highlighting}[]
\CommentTok{#l}
\CommentTok{#isg is a 4-star}
\CommentTok{#m}
\CommentTok{#Yes, a vertex in a tree can have 2 parents. Note, however, that}
\CommentTok{#we generally consider only rooted trees as having parents.}
\CommentTok{#No, a vertex in a rooted tree cannot have 2 parents, since the }
\CommentTok{#parents must go back to the root thus creating a cycle.}
\end{Highlighting}
\end{Shaded}

\end{document}
